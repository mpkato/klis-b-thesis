\begin{abst}
本論文では,順序付けられた一部のエンティティを訓練データ,エンティティの数値属性を特徴として線形関数を学習することで,エンティティのランキングを行う問題に取り組む.
例えば,$f($Entity$) = +0.5($GDP$) − 0.8($自殺者数$)$という関数を「幸福度」によって順序づけられたエンティティから学習する.
この問題では訓練データの数が非常に少ないため,
Web上に存在する大量の「文脈」による補助を受けて学習を行う機械学習手法,
文脈誘導型学習を本論文では提案する.
実験では3クラスのエンティティについて158個の様々な順序によるランキングを行った.
実験結果から,既存のランキング学習手法よりも高い精度が文脈誘導型学習によって得られることが明らかとなった.
本論文では,順序付けられた一部のエンティティを訓練データ,エンティティの数値属性を特徴として線形関数を学習することで,エンティティのランキングを行う問題に取り組む.
例えば,$f($Entity$) = +0.5($GDP$) − 0.8($自殺者数$)$という関数を「幸福度」によって順序づけられたエンティティから学習する.
この問題では訓練データの数が非常に少ないため,
Web上に存在する大量の「文脈」による補助を受けて学習を行う機械学習手法,
文脈誘導型学習を本論文では提案する.
実験では3クラスのエンティティについて158個の様々な順序によるランキングを行った.
実験結果から,既存のランキング学習手法よりも高い精度が文脈誘導型学習によって得られることが明らかとなった.
本論文では,順序付けられた一部のエンティティを訓練データ,エンティティの数値属性を特徴として線形関数を学習することで,エンティティのランキングを行う問題に取り組む.
例えば,$f($Entity$) = +0.5($GDP$) − 0.8($自殺者数$)$という関数を「幸福度」によって順序づけられたエンティティから学習する.
この問題では訓練データの数が非常に少ないため,
Web上に存在する大量の「文脈」による補助を受けて学習を行う機械学習手法,
文脈誘導型学習を本論文では提案する.
実験では3クラスのエンティティについて158個の様々な順序によるランキングを行った.実験結果から,既存のランキング学習手法よりも高い精度が文脈誘導型学習によって得られることが明らかとなった.
実験では3クラスのエンティティについて158個の様々な順序によるランキングを行った.実験結果から,既存のランキング学習手法よりも高い精度が文脈誘導型学習によって得られることが明らかとなった.
実験では3クラスのエンティティについて158個の様々な順序によるランキングを行った.実験結果から,既存のランキング学習手法よりも高い精度が文脈誘導型学習によって得られることが明らかとなった.
実験では3クラスのエンティティについて158個の様々な順序によるランキングを行った.

\end{abst}