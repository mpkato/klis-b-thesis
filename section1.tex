\chapter{緒論}
\label{sq:Introduction}
エンティティの順序は人気の高いコンテンツであり,多くのWebサイトで紹介されている.
例えば,国,都市,企業,有名人,および,製品などが,居住性,革新性,美しさ,性能などのさまざまな基準によって順序づけられている.
順序は比較文
(例えば, ``レオナルド ディカプリオはブラッド ピットよりも背が高い'')や,
最上級(例えば,``東京は最も物価の高い都市である''),
ランキング(例えば,``1位:デンマーク,2位:スイス,3位:アイスランド'')などの形式で表現されている.
人々はそのようなエンティティの順序を意思決定,比較,計画などに利用する.

順序づけの基準の詳細は,場合によっては正確かつ客観的に記述されているが,
省略されたり基準が主観的である場合もある.
例えば,デンマークはいくつかのウェブサイトで最も幸せな国として紹介されているが詳細な証拠が提示されていない場合も多い.
その根拠は,国民の自己申告,他の国の評判,あるいは,
GDPや自殺者数などの客観的尺度かもしれない.
不確かな証拠だけでそのような知識を獲得することは,
偏見やステレオタイプにつながる可能性がある.

本論文では,部分的に観測された順序を訓練データとして,エンティティの数値属性を特徴として使用することで,エンティティの順序を学習する方法を提案する.
学習されたモデルによって,Web上の順序を説明することが可能であり,
また,学習に使用されていない他のエンティティをランキングすることも可能である.
順序付けは数値属性の線形関数$f$によって表すことができると仮定する.
これは,主にユーザへの説明能力が高く,
現実世界における順序の大部分は,いくつかの尺度を線形結合することによって求められるという事実があるためである.
例えば,「幸福度」という「順序基準」によって順序づけられた(デンマーク,スイス,アイスランド)というエンティティ列と
「GDP」,「国土面積」,「自殺者数」という数値属性が与えられた時,
$f($Entity$) = +0.5($GDP$) − 0.8($自殺者数$)$という線形関数を学習することができる.

この問題の主な課題の1つは,訓練データが不足していることである.
多くのWebサイトでは,すべてのエンティティの順序は記述されてはいない.
さらに,一部のエンティティクラスでは,すべてのエンティティの順序が記述されていたとしても訓練データのサイズは十分ではない(たとえば,日本の都道府県数は47であり,アメリカの州の数は50である).
数値属性の数は,通常,Web上で順序づけられたエンティティの数よりもはるかに多いので,エンティティの順序を学習する際に,深刻な過学習の問題が起こりうる.

この本質的な問題に対処するために,我々は{\bf 文脈誘導型学習}と呼ばれる学習方法を提案する.
文脈誘導型学習は,順序づけられたエンティティだけでなく,
各特徴の重みを学習するために順序基準と属性に関する文脈を利用する.
文脈は追加の情報をモデルに与え,過学習を防止して学習を正しい方向に誘導する.
例えば, ``幸福度''という基準に関する線形関数の, ``GDP''という特徴に対する重みは,
(デンマーク,スイス,アイスランド)というエンティティ列だけでなく,
 ``{\bf GDP}は{\bf 幸福}のための重要な要素である''や``{\bf GDP}が向上すれば{\bf 幸福度}も改善する'',といった「幸福」と「GDP」の文脈(「幸福」と「GDP」の関係について記述された文や文書)によっても推測されることになる.
文脈誘導型学習は,文脈に対するアノテーションを必要とせず, 
代わりに,文脈と関数$f$中の重みの関係を学習するために,複数の順序を同時に学習する.
 
我々の知る限りにおいて,
文脈誘導型学習は,機械学習において「ラベル基準」と特徴の文脈を直接活用する最初の試みである(ラベル基準とはラベル付けの基準を表すテキストのことであり,エンティティランキングの問題における順序基準に対応する).
文脈誘導型学習は,ラベル基準と特徴の関係が特定のコーパスに記述されてれば,
順位付けだけでなく分類や回帰の問題にも適用することが可能である.

実験では,国,都道府県,カメラの3つのエンティティクラスを対象とし158種類の順序を学習した.
訓練された順序の精度を測定し,
与えられた順序が学習されたモデルによってどれだけ説明可能であるかを評価した.
実験の結果,文脈誘導型学習は既存のランキング学習手法よりも統計的有意に高い精度を示した.

この論文における我々の貢献を以下に示す:
(1) 数値属性を特徴とした,エンティティの順位付けを学習する問題を提案した.
これによって,Web上の順序を正確に理解し,さまざまな基準でエンティティを順序付けすることが可能である.
(2)文脈誘導型学習を提案した.これは,過学習を防止するためにラベル基準および特徴の文脈を利用する一般的な機械学習モデルである.
(3)幅広い順序に対する実験を行い,
エンティティの順序づけタスクにおいて文脈誘導型学習の有効性を示した.

本論文の構成は以下の通りである.
2節ではエンティティのランキングに関する関連研究,および,
文脈誘導型学習と他の機械学習手法との関係について述べる.
3節では問題設定を説明し,文脈誘導型学習,および.
その主問題への適用方法について述べる.
4節では実験結果を示す.
最後に,5節では今後の課題と共に本論文の結論を述べる.